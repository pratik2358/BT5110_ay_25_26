\documentclass{beamer}
\usepackage[T1]{fontenc}
\usepackage[english]{babel}
\usepackage{lmodern}
\usepackage{amsmath,amssymb}
\usepackage{algorithm}
\usepackage{algorithmic}
\usepackage{graphicx}
\usepackage{subcaption}
\usepackage{caption}
\usepackage{tikz}
\usetikzlibrary{overlay-beamer-styles}
\usepackage{booktabs,multicol,stackengine}
\usepackage[table,dvipsnames]{xcolor}
\usepackage{cancel}
\usepackage{moresize}

\usepackage{hyperref}

\usepackage{listings}
\definecolor{nus-orange}{RGB}{239,124,0}
\definecolor{nus-blue}{RGB}{0,61,124}
\definecolor{halfgray}{gray}{0.55}

\lstset{
  basicstyle=\ttfamily\footnotesize
  keywordstyle=\bfseries\color{nus-blue},
  stringstyle=\color{ForestGreen},
  numbers=left,
  numberstyle=\scriptsize\color{halfgray},
  rulesepcolor=\color{nus-orange},
  frame=shadowbox,
  columns=flexible,
  showstringspaces=false,
  keepspaces=true,
  tabsize=2,
  breaklines=true
}

\renewcommand{\figurename}{Figure}
\renewcommand{\algorithmname}{Algorithm}

\usepackage{SHU}
\input{macros}

\title{BT5110: Tutorial 3 — Functional Dependencies}
\author{\href{https://pratik2358.github.io/}{Pratik Karmakar}}
\institute{School of Computing,\\ National University of Singapore}
\date{AY25/26 S1}

\begin{document}
\begin{frame}
  \titlepage
  \IfFileExists{nus-logo.png}{
    \begin{figure}[htpb]\centering
      \includegraphics[keepaspectratio, scale=0.18]{nus-logo.png}
    \end{figure}
  }{}
\end{frame}
\include{tut_03_example}

\section{Solutions}

\begin{frame}[fragile]{Question 1.a.}
\tiny
Your company, Apasaja Private Limited, is commissioned by \emph{Toko Kopi Luwak} to design the relational schema for coffee beans, drinks, and cafes. We are given
\[
R=\{A,B,C,D,E,F,G,H\},\\
\Sigma=\{\{A\} \to \{C,E\},\{A,B\} \to \{D\},\{F\} \to \{H\},\{C,E\} \to \{A\},\{B,C,E\} \to \{D\},\{A,B,F\} \to \{D,G\},\{B,C,E,F\} \to \{G\}\}.
\]

\pause
\footnotesize
\vspace{0.5em}
\begin{columns}
\begin{column}{0.48\textwidth}
\vspace{0.5em}
\textbf{Attribute mapping}\\[-0.2em]
\begin{tabular}{@{}ll@{}}
\toprule
\textbf{Symbol} & \textbf{Meaning} \\
\midrule
A & \texttt{name} (bean brand) \\
B & \texttt{dname} (drink name) \\
C & \texttt{cultivar} \\
D & \texttt{price} \\
\bottomrule
\end{tabular}
\end{column}
\pause
\begin{column}{0.48\textwidth}
\textbf{Attribute mapping}\\[-0.2em]
\begin{tabular}{@{}ll@{}}
\toprule
\textbf{Symbol} & \textbf{Meaning} \\
\midrule
E & \texttt{region} \\
F & \texttt{bname} (branch) \\
G & \texttt{qty} (quantity sold) \\
H & \texttt{address} \\
\bottomrule
\end{tabular}
\end{column}
\end{columns}

\medskip
\scriptsize
\textcolor{teal}{Notes: A bean is identified by \alert{name} or by \alert{(cultivar, region)}; a drink name is unique \emph{per bean}; branches sell drinks, with address and quantity recorded.}
\end{frame}

\begin{frame}{Question 1.b. \; Candidate keys via closures}
\footnotesize
\textbf{Observation:} Since \alert{$B$} and \alert{$F$} never appear on the RHS of any FD, \alert{every key must contain $B$ and $F$}. We enumerate closures starting from supersets of $\{B,F\}$.

\pause
\tiny
\begin{columns}
\begin{column}{0.32\textwidth}
\textbf{Size 2}\\[-0.1em]
$\{B,F\}^+ = \{B,F,H\}$ \; (not a key)

\medskip
\textbf{Size 3 (supersets of $\{B,F\}$)}\\[-0.3em]
$\alert{\{A,B,F\}^+ = \{A,B,C,D,E,F,G,H\}}$ \\
$\{B,C,F\}^+ = \{B,C,F,H\}$ \\
$\{B,D,F\}^+ = \{B,D,F,H\}$ \\
$\{B,E,F\}^+ = \{B,E,F,H\}$ \\
$\{B,F,G\}^+ = \{B,F,G,H\}$ \\
$\{B,F,H\}^+ = \{B,F,H\}$
\end{column}
\pause
\vspace{0.2em}
\begin{column}{0.34\textwidth}
\textbf{Size 4 (not containing $\{A,B,F\}$)}\\[-0.1em]
$\{B,C,D,F\}^+ = \{B,C,D,F,H\}$ \\
$\alert{\{B,C,E,F\}^+ = \{A,B,C,D,E,F,G,H\}}$ \\
$\{B,C,F,G\}^+ = \{B,C,F,G,H\}$ \\
$\{B,C,F,H\}^+ = \{B,C,F,H\}$ \\
$\{B,D,E,F\}^+ = \{B,D,E,F,H\}$ \\
$\{B,D,F,G\}^+ = \{B,D,F,G,H\}$ \\
$\{B,D,F,H\}^+ = \{B,D,F,H\}$ \\
$\{B,E,F,G\}^+ = \{B,E,F,G,H\}$ \\
$\{B,E,F,H\}^+ = \{B,E,F,H\}$ \\
$\{B,F,G,H\}^+ = \{B,F,G,H\}$
\end{column}
\pause
\begin{column}{0.34\textwidth}
\textbf{Larger supersets} (that are not supersets of keys above)\\
All such closures remain proper subsets of $R$.

\medskip
\textbf{Conclusion}\\
\alert{Candidate keys: $\{A,B,F\}$ and $\{B,C,E,F\}$.}
\end{column}
\end{columns}

\medskip
\scriptsize
\begin{center}
    Tedious by hand?\\
Take a look into this repository: \href{https://github.com/pratik2358/fucntional_dep}{\textcolor{blue}{Here}}
\end{center}
\end{frame}

\begin{frame}{Question 1.c. \; Prime attributes}
\small
Keys are \alert{$\{A,B,F\}$} and \alert{$\{B,C,E,F\}$}. Prime attributes are those that appear in at least one key.

\medskip
\[
\text{Prime attributes} \;=\; \{A,B,F\}\cup\{B,C,E,F\} \;=\; \alert{\{A,B,C,E,F\}}.
\]
\end{frame}

\begin{frame}[fragile]{Question 2.a. \; Minimal cover of $R$ with $\Sigma$}
\footnotesize
\begin{columns}
\begin{column}{0.30\textwidth}
\textbf{Start from $\Sigma$:}\\
\(
\{A\} \to \{C,E\},\\
\{A,B\} \to \{D\},\\
\{F\} \to \{H\},\\
\{C,E\} \to \{A\},\\
\{B,C,E\} \to \{D\},\\
\{A,B,F\} \to \{D,G\},\\
\{B,C,E,F\} \to \{G\}.
\)
\end{column}

\pause
\begin{column}{0.30\textwidth}
\textcolor{teal}{Step 1: Split RHS}\\
\(
\{A\} \to \{C\},\\
\{A\} \to \{E\},\\
\{A,B\} \to \{D\},\\
\{F\} \to \{H\},\\
\{C,E\} \to \{A\},\\
\{B,C,E\} \to \{D\},\\
\{A,B,F\} \to \{D\},\\
\{A,B,F\} \to \{G\},\\
\{B,C,E,F\} \to \{G\}.
\)
\end{column}
\pause
\begin{column}{0.40\textwidth}
\textcolor{blue}{Step 2: Reduce LHS where possible}\\
\(
\{A\} \to \{C\},\\
\{A\} \to \{E\},\\
\{A,B\} \to \{D\},\\
\{F\} \to \{H\},\\
\{C,E\} \to \{A\},\\
\{B,C,E\} \to \{D\},\\
\cancel{\textcolor{gray}{\{A,B,F\} \to \{D\}}}\text{ because } \{A,B\} \to \{D\}\\
\{A,B,F\} \to \{G\},\\
\{B,C,E,F\} \to \{G\}.
\)
\end{column}
\end{columns}
\end{frame}
\begin{frame}[fragile]{Question 2.a. \; Minimal cover of $R$ with $\Sigma$}
\begin{columns}
    \begin{column}{0.5\textwidth}
        \textcolor{teal}{Step 3: Remove redundancies}\\
        \(
\{A\} \to \{C\},\\
\{A\} \to \{E\},\\
\cancel{\textcolor{gray}{\{A,B\} \to \{D\}}}\tiny\text{ because }\{A\}\to\{C,E\} \text{ and } \{B,C,E\}\to\{D\},\\
\normalsize
\{F\} \to \{H\},\\
\{C,E\} \to \{A\},\\
\{B,C,E\} \to \{D\},\\
\cancel{\textcolor{gray}{\{A,B,F\} \to \{G\}}}\tiny\text{ because }\{A\}\to\{C,E\} \text{ and } \{B,C,E\}\to\{D\},\\
\normalsize
\{B,C,E,F\} \to \{G\}.
\)
    \end{column}
    \begin{column}{0.5\textwidth}
        \textcolor{red}{Minimal Cover}\\
\(
\{A\} \to \{C\},\\
\{A\} \to \{E\},\\ 
\{F\} \to \{H\},\\ 
\{C,E\} \to \{A\},\\
\{B,C,E\} \to \{D\},\\ 
\{B,C,E,F\} \to \{G\} .
\)
    \end{column}
\end{columns}
\medskip
\scriptsize
\pause
(Alternative minimal covers exist, e.g., replace $\{B,C,E\} \to \{D\}$ by $\{A,B\} \to \{D\}$ and $\{B,C,E,F\} \to \{G\}$ by $\{A,B,F\} \to \{G\}$.) \hfill
\end{frame}

\begin{frame}[fragile]{Question 2.b. \; Canonical cover (merged by LHS)}
\small
One canonical cover (grouping by identical LHS) is:\\
\(
\{A\} \to \{C,E\},\\ 
\{F\} \to \{H\},\\ 
\{C,E\} \to \{A\},\\ 
\{B,C,E\} \to \{D\},\\ 
\{B,C,E,F\} \to \{G\}
\)

\medskip
\scriptsize
\pause
Other canonical covers are possible depending on which equivalent minimal cover you merge.
\end{frame}

\begin{frame}
\begin{center}
Questions?\\
Drop a mail at: pratik.karmakar@u.nus.edu
\end{center}
\end{frame}
\end{document}